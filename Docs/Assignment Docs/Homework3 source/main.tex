%\documentclass[draft, 11pt, a4paper]{article}  %compiles faster for editing
\documentclass[11pt, a4paper]{article}
%\usepackage{geometry}
\usepackage[inner=1.5cm,outer=1.5cm,top=2.5cm,bottom=2.5cm]{geometry}
\pagestyle{empty}
\usepackage{graphicx}
\usepackage{fancyhdr, lastpage, bbding, pmboxdraw}
\usepackage[usenames,dvipsnames]{color}
\usepackage{epigraph}

\newcommand{\hilight}[1]{\colorbox{BurntOrange}{#1}}  % changes will be highlighted in BurntOrange

% Set links, etc. to different colors
\definecolor{OliveGreen}{cmyk}{0.64,0,0.95,0.40}
\definecolor{CadetBlue}{cmyk}{0.62,0.57,0.23,0}
\definecolor{lightlightgray}{gray}{0.93}
\definecolor{darkblue}{rgb}{0,0,.6}
\definecolor{darkred}{rgb}{.7,0,0}
\definecolor{darkgreen}{rgb}{0,.6,0}
\definecolor{red}{rgb}{.98,0,0}
\usepackage[bookmarksopen, backref,colorlinks,pagebackref,pdfusetitle,urlcolor=darkblue,citecolor=darkblue,linkcolor=darkred,bookmarksnumbered,plainpages=false]{hyperref}


\renewcommand{\thefootnote}{\fnsymbol{footnote}}

\pagestyle{fancyplain}
\fancyhf{}
\lhead{ \fancyplain{}{Product Design Specifications} }
%\chead{ \fancyplain{}{} }
%\rhead{ \fancyplain{}{} }
%\rfoot{\fancyplain{}{page \thepage\ of \pageref{LastPage}}}
\fancyfoot[R] {\thepage}     %use this instead
\thispagestyle{plain}

%%%%%%%%%%%% LISTING %%%
\usepackage{listings}
\usepackage{caption}
\DeclareCaptionFont{white}{\color{white}}
\DeclareCaptionFormat{listing}{\colorbox{gray}{\parbox{\textwidth}{#1#2#3}}}
\captionsetup[lstlisting]{format=listing,labelfont=white,textfont=white}
\usepackage{fancyvrb}
\usepackage{acronym}
\usepackage{amsthm}

%%% My packages:
\usepackage{graphicx} %package to manage images
\graphicspath{ {images/} }  %set default image path
\usepackage{chngcntr}
\usepackage{float}       % useful for positioning figures / etc.
\counterwithin{table}{section}      % counts fig 1.1... 1.2... etc
\counterwithin{figure}{section}     % same for figures.. for List of Figures page

% for highlighting text
\usepackage{soul}
\newcommand{\hlc}[2][yellow]{ {\sethlcolor{#1} \hl{#2}} }  %Not sure why I set it to yellow.. maybe used for something else besides changes
\usepackage[export]{adjustbox} %for left/right aligned stuff
\usepackage{array} % for centering things in tables
\newcolumntype{M}[1]{>{\centering\arraybackslash}m{#1}} %SAME AS ABOVE

\lstset{
%language=bash,                          % Code language
basicstyle=\ttfamily,                   % Code font, Examples: \footnotesize, \ttfamily
keywordstyle=\color{OliveGreen},        % Keywords font ('*' = uppercase)
commentstyle=\color{gray},              % Comments font
numbers=left,                           % Line nums position
numberstyle=\tiny,                      % Line-numbers fonts
stepnumber=1,                           % Step between two line-numbers
numbersep=5pt,                          % How far are line-numbers from code
backgroundcolor=\color{lightlightgray}, % Choose background color
frame=none,                             % A frame around the code
tabsize=2,                              % Default tab size
captionpos=t,                           % Caption-position = bottom
breaklines=true,                        % Automatic line breaking?
breakatwhitespace=false,                % Automatic breaks only at whitespace?
showspaces=false,                       % Dont make spaces visible
showtabs=false,                         % Dont make tabls visible
columns=flexible,                       % Column format
morekeywords={__global__, __device__},  % CUDA specific keywords
}

%%%%%%%%%%%%%%%%%%%%%%%%%%%%%%%%%%%%

\makeatletter                                   %fix for figure names and numbers overlapping
\def\l@figure{\@dottedtocline{1}{1.5em}{4.5em}}   % in the List Of Figures
\makeatother



\fancyfoot[L]{\small{\textit{Revision 1.2}}} %revision number here starting on TOC

\begin{document}

\pagenumbering{Roman} %roman numerals for title.. TOC.. list of figures/tables..
\thispagestyle{empty}   % removes page numbering for title page

\begin{figure}[t]
    \begin{center}
    {\huge \textsc{ Product Design Specifications}}
    \end{center}

    \begin{center}
    \LARGE Inductive Charging
    \end{center}

    \begin{center}
    \normalsize \textit{Practicum Team 10}
    \end{center}
\end{figure}

\vspace*{\fill}

\begingroup
    \begin{center}
    \includegraphics[scale=3.5]{MCECS.jpg}
    \end{center}
\endgroup
    
\vspace*{\fill}
    
\begin{figure}[b]
    \includegraphics[width=.3\textwidth,center]{pdx.png}
\end{figure}


% Contact info footer
    \begin{figure}[!b]
    \begin{center}
    \rule{6in}{0.4pt}
    \begin{minipage}[t]{.75\textwidth}
    \begin{tabular}{llcccll}
    \textbf{Team Members:}  Tyler Gilbert, Kati Dahn, Jann Messer, Jeff Brown & & & &\\
    \textbf{Project repository:} \href{https://github.com/tjgilbert/ece411}{https://github.com/tjgilbert/ece411} \\
    \end{tabular}
    \end{minipage}
    \rule{6in}{0.4pt}
    \end{center}
    \end{figure}
    
\cleardoublepage
\phantomsection

\hypertarget{MyToc}{} %For linking back to the TOC
\rhead{ \fancyplain{}{Table of Contents} }
\begin{center}
{\Large \textsc{ Table of Contents}}
\end{center}
\renewcommand{\contentsname}{}
\tableofcontents

\cleardoublepage
\pagenumbering{arabic}   %  Then switch to arabic page numbering at 1. 
\fancyfoot[R] {page \thepage\ of \pageref*{LastPage} }  




\section{Introduction}
\rhead{ \fancyplain{}{Introduction/Requirements} }

    \subsection{Objective}
    This project will address a desire for wireless charging unit for a mobile phone or similar device. The charging unit will be portable and intuitive to use \textit{(portable and intuitive are defined further in the Requirements section)}. The basic use of the product requires the user to place their device on the charging unit pad. There will be no user inputs other than placing their device on the charging station.

    \subsection{Needs Statement}
     The need for this technology is sufficed by the convenient enhancements upon wired chargers. Wireless chargers are more convenient, because they negate the requirement of plugging and unplugging your device. The technology can also help people with physical impairments that are unable to manipulate traditional charging cords.
     Wireless charging technology is also inherently more robust than a physical connector. Charging cords and receptacles can wear out or be damaged by careless insertion.


\section{Requirements}

The purpose of the requirements document is to outline the direction of this project within the requirements of the \textit{ECE411} practicum. All of the requirements weighted \textit{Must} need to be satisfied in order to qualify as a minimum viable product. Requirements weighted \textit{Should} specify desired features, which should be realized before the products' final release, but are not necessary for the minimum function.

    \subsection{Functional Requirements}
        \begin{centering}
        \begin{tabular}{|l|m{7cm}|l|m{7cm}|} \hline
        \textit{\textbf{Req\#}}	& \textit{\textbf{Requirement}} &\textit{\textbf{Weight}}&\textit{\textbf{Justification}} \\ \hline
        1.1	& Have one or more sensors and actuators & Must	& Course requirement \\ \hline
        1.2	& Have a digital or analog processor & Must	& Course requirement \\ \hline
        1.3	& Wirelessly charge a USB device & Must	& This is the premise of the project, this is it's sole purpose \\ \hline
        1.4	& Turn on transmitter when device to be charged is placed in range	& Should & This will save energy because the product will not transmit energy if no device is present \\ \hline
        1.5 & Turn off transmitter when device is removed	& Should & Same as 1.4 \\ \hline
        \end{tabular}
        \end{centering}
    
    \subsection{Performance Requirements}
        \begin{centering}
        \begin{tabular}{|l|m{7cm}|l|m{7cm}|} \hline
        \textit{\textbf{Req\#}}	& \textit{\textbf{Requirement}} &\textit{\textbf{Weight}}&\textit{\textbf{Justification}} \\ \hline
        2.1	& Provide USB spec power (4.75-5.25V, 500mA) \cite{USB20}	& Should & Meeting this spec would guarantee it would work on most mobile devices \\ \hline
        2.2	& Can fit charging unit into a backpack (300mm x 300mm x 150mm)	& Should & Meeting these dimensions would qualify the charging unit as portable \\ \hline
        2.3	& Receiver is smaller or equal to 138mm x 67mm	& Should & This is the dimension of the iPhone 7, meeting this footprint would make it compatible with more devices \\ \hline
        \end{tabular}
        \end{centering}
    
\newpage
\rhead{ \fancyplain{}{Requirements} }  %here as a hack to fix formatting


    \subsection{Economic Requirements}
        \begin{centering}
        \begin{tabular}{|l|m{7cm}|l|m{7cm}|} \hline
        \textit{\textbf{Req\#}}	& \textit{\textbf{Requirement}} &\textit{\textbf{Weight}}&\textit{\textbf{Justification}} \\ \hline
        3.1 & Prototypes not to exceed \$75 each & Should & We are not made of money  \\ \hline
        3.2 & Product components not to exceed \$20 ea for quantity 1000 & Should & We are not paying our engineers, or technicians, If we sell the product for \$20+S\&H we \textit{might} turn a profit \\ \hline
        \end{tabular}
        \end{centering}    
        

    \subsection{Health and Safety Requirements}
        \begin{centering}
        \begin{tabular}{|l|m{7cm}|l|m{7cm}|} \hline
        \textit{\textbf{Req\#}}	& \textit{\textbf{Requirement}} &\textit{\textbf{Weight}}&\textit{\textbf{Justification}} \\ \hline
        4.1	& Shut off if base draws excessive current (current rating of Tx coil wire)	& Must &This will serve as short circuit protection, which could lead to damage of product, and/or fire \\ \hline
        4.2	& Shut off if the base unit overheats \mbox{(exceeds $35^{\circ}C$)} \cite{appletemp} & Should & $35^{\circ}C$ is the maximum operating temperature of the iPhone \\ \hline
        \end{tabular}
        \end{centering}
    
    \subsection{Legal Requirements}
        \begin{centering}
        \begin{tabular}{|l|m{7cm}|l|m{7cm}|} \hline
        \textit{\textbf{Req\#}}	& \textit{\textbf{Requirement}} &\textit{\textbf{Weight}}&\textit{\textbf{Justification}} \\ \hline
        6.1	& Code and hardware design will be open source or generated by us& Must	&To preserve our academic honesty and integrity \\ \hline
        6.2	& Any borrowed or modified code/designs will be credited according to the applicable license & Must	&It's the right thing to do\\ \hline
        \end{tabular}
        \end{centering}
    
    
    \subsection{Environmental Requirements}
        \begin{centering}
        \begin{tabular}{|l|m{7cm}|l|m{7cm}|} \hline
        \textit{\textbf{Req\#}}	& \textit{\textbf{Requirement}} &\textit{\textbf{Weight}}&\textit{\textbf{Justification}} \\ \hline
        7.1	& Charging unit compliant with IP22	\cite{iprating} & Should & Users may spill liquids on the base unit, but as most mobile devices are not water resistant the receive units will not carry this rating \\ \hline
        \end{tabular}
        \end{centering}
    
    
    \subsection{Usability Requirements}
        \begin{centering}
        \begin{tabular}{|l|m{7cm}|l|m{7cm}|}        \hline
        \textit{\textbf{Req\#}}	& \textit{\textbf{Requirement}} &\textit{\textbf{Weight}}&\textit{\textbf{Justification}} \\ \hline
        8.1 & Orientation/location for device placement on charging unit is marked & Should & Marking the location for optimal charging will reduce the time to full charge\\ \hline
        8.2 & User able set up within 5 minutes without user manual by a college student & Should & This is a metric to decide if the device is user friendly \\ \hline
        8.3 & Base station will indicate when device is coupled/charging & Should & User feedback will increase confidence that the product is operating correctly                                \\ \hline
        8.4 & Base station will indicate when device is done charging & May& Further user feedback on charge state would would be a more obvious indication than having to check the phone \\ \hline
        \end{tabular}
        \end{centering}

\newpage   


\phantomsection
\addcontentsline{toc}{section}{References} %References don't show up in TOC on their own

\begin{thebibliography}{}

\bibitem{lecture}
Faust, Mark. "Requirements Specification" ECE 411 Industry Design Practices. Portland State University, Portland, OR. 18 October 2016. Lecture/PowerPoint


\bibitem{USB20}
"Universal Serial Bus Specification Revision 2.0"
\emph{Compaq Computer Corporation,
Hewlett-Packard Company, Intel Corporation, Lucent Technologies Inc,
Microsoft Corporation, NEC Corporation, Koninklijke Philips Electronics N.V.} \small{[\href{http://www.pjrc.com/teensy/beta/usb20.pdf\#page=206}{PDF DOCUMENT}]}, \normalsize 2000
 
\bibitem{appletemp}
"Keeping IPhone, IPad, and IPod Touch within Acceptable Operating Temperatures."
\emph{Apple Support.} N.p., 28 Sept. 2016. Web. 24 Oct. 2016. \url{https://support.apple.com/en-us/HT201678}

\bibitem{iprating}
"IP Rated Enclosures Explained"
\emph{The Enclosure Company (International) Ltd} N.p., n.d. Web. 23 Oct. 2016. \url{http://www.enclosurecompany.com/ip-ratings-explained.php}

\bibitem{forbes}
"How Often Do Americans Upgrade Their Smartphones?" \emph{Forbes.} Forbes Magazine, 9 July, 2015. Web. 24 Oct. 2016.


\rhead{ \fancyplain{}{References} }

\end{thebibliography}
% that's all folks







    

    


    

    
    

    
    



    

    

    

    
    

\end{document}
	
	